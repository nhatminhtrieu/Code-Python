\documentclass{article}
\usepackage[a4paper,top=3cm,bottom=2.5cm,left=2.5cm,
            right=2cm,marginparwidth=1.75cm,
            headheight=5pt]{geometry}
\usepackage[T5]{fontenc}
\usepackage[utf8]{inputenc}
\usepackage[document]{ragged2e}
\usepackage[vietnamese]{babel}
\usepackage[unicode]{hyperref}
\usepackage{amsmath}
\usepackage{setspace}
\usepackage{graphicx}
\usepackage{caption}
\usepackage{subcaption}
\usepackage{tcolorbox}
\usepackage{listings}
\usepackage{hyperref}
\usepackage{xcolor}
\usepackage{titlesec}
\usepackage{floatrow}
\usepackage[nottoc]{tocbibind}
\usepackage{mdframed}
\usepackage{amsmath}
\usepackage{amssymb}
\usepackage{tgbonum}
\usepackage{type1cm}
\usepackage{indentfirst}
\usepackage{lettrine}
\usepackage{colortbl}
\usepackage{fancyhdr}
\usepackage{wrapfig}
\usepackage{lastpage}
\usepackage{url}
\addto\captionsenglish{
  \renewcommand{\contentsname}{MỤC LỤC}%
  \renewcommand{\listfigurename}{Danh sách ảnh}%
  \renewcommand{\listtablename}{Danh sách bảng}%
  \renewcommand{\figurename}{Hình}
  \renewcommand{\tablename}{Bảng}
}
\pagestyle{fancy}
\fancyhf{}
\rhead{Toán ứng dụng và thống kê}
\lhead{\color{cyan}Đồ án 2: Image Processing}
\lfoot{Trang \thepage /\pageref{LastPage}}
\renewcommand{\footrulewidth}{0.4pt}
\setlength{\parindent}{5pt}
\setlength{\parskip}{1cm}
\renewcommand{\baselinestretch}{1.5}
\newmdenv[linecolor=black,skipabove=\topsep,skipbelow=\topsep,
leftmargin=2.5cm,rightmargin=2.5cm,
innerleftmargin=5cm,innerrightmargin=5cm]{mybox}
\usepackage{multicol}
\usepackage{indentfirst}
\usepackage{color}
\usepackage{apacite}
\usepackage{tikz}
\graphicspath{{Figures/}} 
\usepackage[square, numbers, comma, sort&compress]{natbib}  
\usepackage{lipsum}
\usetikzlibrary{calc}
\setlength{\columnseprule}{2pt}
\def\columnseprulecolor{\color{black}}
\def\maru#1{\textcircled{\scriptsize#1}}

\begin{document}
% Bìa trang
\begin{titlepage}
\begin{tikzpicture}[remember picture,overlay,inner sep=0,outer sep=0]
     \draw[blue!70!black,line width=4pt] ([xshift=-1.5cm,yshift=-2cm]current page.north east) coordinate (A)--([xshift=2cm,yshift=-2cm]current page.north west) coordinate(B)--([xshift=2cm,yshift=2cm]current page.south west) coordinate (C)--([xshift=-1.5cm,yshift=2cm]current page.south east) coordinate(D)--cycle;

     \draw ([yshift=0.5cm,xshift=-0.5cm]A)-- ([yshift=0.5cm,xshift=0.5cm]B)--
     ([yshift=-0.5cm,xshift=0.5cm]B) --([yshift=-0.5cm,xshift=-0.5cm]B)--([yshift=0.5cm,xshift=-0.5cm]C)--([yshift=0.5cm,xshift=0.5cm]C)--([yshift=-0.5cm,xshift=0.5cm]C)-- ([yshift=-0.5cm,xshift=-0.5cm]D)--([yshift=0.5cm,xshift=-0.5cm]D)--([yshift=0.5cm,xshift=0.5cm]D)--([yshift=-0.5cm,xshift=0.5cm]A)--([yshift=-0.5cm,xshift=-0.5cm]A)--([yshift=0.5cm,xshift=-0.5cm]A);


     \draw ([yshift=-0.3cm,xshift=0.3cm]A)-- ([yshift=-0.3cm,xshift=-0.3cm]B)--
     ([yshift=0.3cm,xshift=-0.3cm]B) --([yshift=0.3cm,xshift=0.3cm]B)--([yshift=-0.3cm,xshift=0.3cm]C)--([yshift=-0.3cm,xshift=-0.3cm]C)--([yshift=0.3cm,xshift=-0.3cm]C)-- ([yshift=0.3cm,xshift=0.3cm]D)--([yshift=-0.3cm,xshift=0.3cm]D)--([yshift=-0.3cm,xshift=-0.3cm]D)--([yshift=0.3cm,xshift=-0.3cm]A)--([yshift=0.3cm,xshift=0.3cm]A)--([yshift=-0.3cm,xshift=0.3cm]A);

   \end{tikzpicture}
\newcommand{\HRule}{\rule{\linewidth}{0.5mm}}
\center

\textsc{\Large TRƯỜNG ĐẠI HỌC KHOA HỌC TỰ NHIÊN}\\[0.5cm]
\textsc{\Large KHOA CÔNG NGHỆ THÔNG TIN}\\[1cm]
\includegraphics[width=0.3\textwidth]{logo/KHTN.jpg}\\[1cm]

\HRule \\[0.4cm]
{\huge \bfseries ĐỒ ÁN 2: IMAGE PROCESSING} \\[0.4cm]
{\large TOÁN ỨNG DỤNG VÀ THỐNG KÊ}\\[0.1cm]
\HRule \\[1.5cm]

\centerline{\Large{\textbf{Triệu Nhật Minh — 21127112 — 21CLC02}}}
\vspace{2.5cm}
\centerline{\large{\textit{Giảng viên hướng dẫn}}}
\vspace{0.25cm}
\centerline{\large{Vũ Quốc Hoàng}}
\centerline{\large{Lê Thanh Tùng}}
\centerline{\large{Nguyễn Văn Quang Huy}}
\centerline{\large{Phan Thị Phương Uyên}}

\vspace{2.5cm}
\centerline{\today}


\vfill % Wipe blank space of the page.
\end{titlepage}

% Mục lục tự động
\setlength{\parskip}{.5em}
\tableofcontents
\newpage

% Table of Figures & Tables
\setlength{\parskip}{.5em}
%\listoffigures
%\listoftables
\newpage

% Bắt đầu nội dung
\section{Hướng dẫn sử dụng}
\begin{description}
  \item[Bước 1:] Run all các cell trong notebook \textbf{21127112.ipynb} (hoặc chỉ run cell cuối cùng để chạy hàm main nếu đã khởi chạy các hàm bổ trợ từ trước).
  \item[Bước 2: ] Nhập đường dẫn tới ảnh cần nén (chỉ cần tên ảnh nếu ảnh đã nằm cùng thư mục với notebook).
  \item[Bước 3: ] 
\end{description}

\section{Ý tưởng thực hiện}
\begin{description}
  \item[Bước 1: ] 
\end{description}

\section{Mô tả}
% Template mô tả hàm
% \textbf{Input:} \\
% \textbf{Output:}

% \paragraph{Mô tả:}
\subsection{Nhóm hàm chính}

\subsection{Nhom hàm bổ trợ}
\pagebreak
\newpage
\subsection{Nhận xét}
\subsubsection{Về ảnh đầu ra}

\subsubsection{Về tài nguyên sử dụng}

\centerline{\includegraphics[width=0.2\textwidth]{image/performance.png}}
\newpage
\section{Tài liệu tham khảo}
\begin{itemize}
    \item \href{https://youtu.be/uLs-EYUpGAw?t=231}{How to Code K-Means in Python (No Sklearn)}
    \item \href{https://numpy.org/doc/stable/reference/generated/numpy.reshape.html}{numpy.reshape}
    \item \href{https://numpy.org/doc/stable/reference/random/generated/numpy.random.randint.html}{numpy.random.randint}
    \item \href{https://numpy.org/doc/stable/reference/random/generated/numpy.random.choice.html}{numpy.random.choice}
    \item \href{https://numpy.org/doc/stable/reference/generated/numpy.sum.html}{numpy.sum}
    \item \href{https://numpy.org/doc/stable/reference/generated/numpy.argmin.html}{numpy.argmin}
    \item \href{https://numpy.org/doc/stable/reference/generated/numpy.mean.html}{numpy.mean}
    \item \href{https://numpy.org/doc/stable/reference/generated/numpy.array_equal.html}{numpy.array\_equal}
    \item \href{https://matplotlib.org/stable/api/_as_gen/matplotlib.pyplot.figure.html}{matplotlib.pyplot.figure}
    \item \href{https://www.reddit.com/r/learnpython/comments/4g4dru/error_converting_images_to_pdf/}{Error converting images to pdf}
    \item \href{https://numpy.org/doc/stable/user/basics.broadcasting.html}{Broadcasting explained}
\end{itemize}

\end{document}